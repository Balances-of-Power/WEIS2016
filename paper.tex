%& -shell-escape
%& -shell-escape

\documentclass{sig-alternate}
\usepackage[colorlinks=true,urlcolor=black,linkcolor=cyan,citecolor=cyan]{hyperref}
\usepackage[hyphenbreaks]{breakurl}

\pagenumbering{arabic}

%\usepackage{amsthm}

\usepackage{tikz}
\usepackage{pgfplots}
\usepackage[latin1]{inputenc}

\usepackage{url}
\usepackage{times}
\usepackage{graphicx}
\usepackage{amsmath}
\usepackage{amssymb}

\usepackage{latexsym}
\usepackage{enumerate}
\usepackage{color}
\usepackage{booktabs}

\usepackage{pifont}
\usepackage[nocompress]{cite}

\usepackage{dcolumn}
\newcolumntype{d}[1]{D{.}{.}{#1}}
\newcolumntype{.}{D{.}{.}{-1}}

\newcommand{\urlBibTeX}[1]{\url{#1}} %%% needed for URL to print nice


\newtheorem{theorem}{Theorem}[section]
\newtheorem{lemma}[theorem]{Lemma}
\newtheorem{proposition}[theorem]{Proposition}
\newtheorem{corollary}[theorem]{Corollary}

%\newenvironment{proof}[1][Proof]{\begin{trivlist}
%\item[\hskip \labelsep {\bfseries #1}]}{\end{trivlist}}
\newenvironment{definition}[1][Definition]{\begin{trivlist}
\item[\hskip \labelsep {\bfseries #1}]}{\end{trivlist}}
\newenvironment{example}[1][Example]{\begin{trivlist}
\item[\hskip \labelsep {\bfseries #1}]}{\end{trivlist}}
\newenvironment{remark}[1][Remark]{\begin{trivlist}
\item[\hskip \labelsep {\bfseries #1}]}{\end{trivlist}}
\newcommand{\nicolasc}{\ding{110}\ding{43}\textcolor{blue}}
\newcommand{\benjamin}{\ding{110}\ding{43}\textcolor{red}}
\newcommand{\paul}{\ding{110}\ding{43}\textcolor{green}}

\def\firm#1{Firm~#1}
\def\good#1{good~#1}
\def\buy#1{buy-#1}
\def\decline#1{decline-#1}

%\newcommand{\qed}{\nobreak \ifvmode \relax \else
%      \ifdim\lastskip<1.5em \hskip-\lastskip
%      \hskip1.5em plus0em minus0.5em \fi \nobreak
%      \vrule height0.75em width0.5em depth0.25em\fi}




\begin{document}
\title{Privacy, Polarization, and Passage of Divisive Laws}%\thanks{This research was supported in part by the National Science Foundation through award CCF-0424422 (TRUST - Team for Research in Ubiquitous Secure Technology).}}

%\author{Anonymous submission}
%\date{}

\numberofauthors{2} %  in this sample file, there are a *total*
% of EIGHT authors. SIX appear on the 'first-page' (for formatting
% reasons) and the remaining two appear in the \additionalauthors section.
%
\author{
% You can go ahead and credit any number of authors here,
% e.g. one 'row of three' or two rows (consisting of one row of three
% and a second row of one, two or three).
%
% The command \alignauthor (no curly braces needed) should
% precede each author name, affiliation/snail-mail address and
% e-mail address. Additionally, tag each line of
% affiliation/address with \affaddr, and tag the
% e-mail address with \email.
%
% 1st. author
\alignauthor
Benjamin Johnson\\
       %\affaddr{UC Berkeley}\\
       \email{benjaminejohnson@gmail.com}
% 2nd. author
\alignauthor
Paul Laskowski\\
       %\affaddr{UC Berkeley}\\
       \email{paul@ischool.berkeley.edu}
}


\maketitle

\begin{abstract}
%!TEX root = paper.tex
%abstract.tex
%

An abstract
\end{abstract}
%\mynormality{1.2}
%\newpage
%!TEX root = paper.tex
%intro.tex
%

\section{Introduction}
\label{sec:intro}




In 2006, Utah police received an anonymous tip about drugs being sold out of a house in South Salt Lake. Detective Douglas Fackrell spent several hours surveilling the house in an unmarked car, but saw only modestly suspicious activity.  After a week, he stopped an individual, Edward Strieff, as he was exiting the house and asked for identification.  When he discovered that Strieff had an outstanding warrant for a minor traffic violation, Detective Fackrell proceeded to search him and found methamphetamine in Strieff's pockets.  

This event marked the start of legal battle that reached the US Supreme Court this year.  Although a warrant generally provides adequate reason for a search, the police conceded that Fackrell did not have reasonable suspicion to detain Strieff in the first place.  The supreme court has long held that evidence gained in violation of the Fourth Amendment is inadmissible as evidence - the so-called exclusionary rule.  The case revolves around the extent to which the exclusionary rule protects the privacy of citizens when new information, such as an arrest warrant, arrives after an improper detainment.

Utah v. Strieff reveals the extent to which our value of privacy is bound up in notions of power and polarization.  During oral argument in February, Justice Sonya Sotomayor posed the following question. 

\begin{quote}``What stops us from becoming a police state and just having the police stand on the corner down here and stop every person, ask them for identification, put it through, and if a warrant comes up, searching them?"
\end{quote}

At the heart of this argument is a recognition that privacy protects entire groups of people.  One doesn't have to have drugs in one's pocket to object to arbitrary searches by police.  Privacy places limitations on police power, affecting the playing field faced by all citizens.  Even when the vast majority of police officers abide by strict ethical standards, the prospect of running into a corrupt one remains threatening.  Furthermore, the ability to invade the privacy of citizens has been argued to increase incentives for governments to abuse their power.~\cite{laskowskigovernment}

Calls for privacy are particularly acute when a particular group is disadvantaged or marginalized.  In this case, the laws that police enforce may disproportionally affect the marginalized group.  A classic example is the disparity between sentences for powder cocaine, typically associated with rich white communities, and crack cocaine, which is associated with disadvantaged black communities.  Before the fair sentencing act of 2010, the weight of powder cocaine needed to trigger certain federal criminal penalties was 100 times greater than the weight of crack cocaine that would trigger the same penalties.  This disparity is said to be a significant factor behind the large number of African Americans that were sentenced for drug offenses.~\cite{beaver2009getting}

To understand how laws and police enforcement affect disadvantaged groups, we must also understand how society is polarized between different groups of people to begin with.  How are laws passed that benefit one group at the expense of another group.  Moreover, can privacy protection help marginalized groups overcome their disadvantageous position?

In this paper, we use game theoretic modeling to explore the connections between privacy, polarization, and the passage of divisive laws.  Our framework is based on a population of citizens that influence what laws are passed, or what laws are maintained.  A law is defined in terms of how it impacts each individual, and our model is flexible in that it allows any set of effects.  We define a notion of divisiveness which allows us to measure the extent to which a law disproportionately affects different groups of citizens.

Divisiveness is not the only factor to consider when evaluating laws.  A divisive law may still be justified if it significantly improves welfare.  Progressive taxation is one example in which a law targets groups differently with the frequent aim of enhancing welfare.  On the other hand, some laws may not be divisive at all, but may still be welfare-decreasing or unjust for other reasons.  Nevertheless, we believe that divisiveness should generally be viewed as a cause for concern, especially when a law targets a marginalized group.

Our framework allows citizens to form opinions based on how a law impacts them directly, but it optionally allows them to consider the impact on others as well.  This is achieved through an influence matrix that is multiplied by the direct effect of the law.  This can be used to represent a concern for friends, loyalty to a larger group, or learning from a small number of influential personalities.  The influence matrix also allows us to discuss how polarized society is.  At the end of our analysis, we construct a matrix to model the case of a society with one majority group and one minority group.

Our model assumes that laws that are supported by a majority of citizens are passed or maintained.  Although the democratic process involves many factors before gathering a simple majority, we believe this is a useful and tractable way to explain the types of laws that exist in society.

Using our model of how laws are enforced, we are able to identity four distinct notions of privacy.  Two of these are technological, including strategies that citizens can take to hide features and behaviors from authorities.  The other two are legal notions, depending on a judicial branch that functions as a check on enforcement procedures. %police behavior. 
We describe the function of each of these privacy notions using our two-population model of society.  We find that each type of privacy allows a different set of laws to be passed and enforced, resulting in different effects on divisiveness.  Our work supports the idea that privacy, while far from a perfect cure, has a role to play in mitigating the divisive effects of laws in a polarized society.

%Discussion examples
%\begin{itemize}
%\item Fergeuson style searches
%\item stop and frisk
%\item prohibition
%\item iphone data
%\item Muslim databases
%\end{itemize}



%!TEX root = paper.tex
%related.tex

\section{Related work}
\label{sec:related}

Related work

Privacy on the ground. Hard to find citation in bibtex. Can cite a related journal article instead?

Caviar and Yachts \cite{johnsoncaviar}

Government Surveillance \cite{laskowskigovernment}


%!TEX root = paper.tex
%model.tex

\section{Model}
\label{sec:model} 


%\begin{tikzpicture}[scale = 6]
%
%       % x axis
%\draw [->] (0, 0) -- (1.1, 0) node [below right] {$P_0$};
%\draw [shift={(1.0, 0)}] (0,0.02) -- (0, -0.02) node[below] {$Cr(0)$};
%
%       % y axis
%\draw [->] (0, 0) -- (0, 1.3) node [above left] {$P_1$};
%\draw [shift={(0, 1.2)}] (-0.02,0) -- (0.02,0) node[shift={(-0.8,0)}] {$Cr(1)$};
%
%	%diagonal
%\draw [->,dashed] (-0.1, -0.1) -- (1.2, 1.2) node [above] {$B$};
%
%	%right
%\draw [-,dashed] (1, 0) -- (1, 1.2);
%
%	%top
%\draw [-,dashed] (0, 1.2) -- (1, 1.2); 
%
%	%point A
%\draw[shift={(0.2, 1.2)}] node [above] {$A$};
%
%\end{tikzpicture}

%
%\usetikzlibrary{patterns}
%
%\newcommand{\myaxes}{
%	\draw [->|] (0,0) node [left=1mm] {\scriptsize $$} -- node [rotate=90,above] {vertical label} (0,1.2) node [left=1mm,overlay] {\scriptsize $Cr(1)$};
%	\draw [->|] (0,0) node [below=1mm] {\scriptsize $0$} -- node [below] {horizontal label} (1,0) node [below=1mm] {\scriptsize $Cr(0)$};
%}
%
%\tikzstyle{full-protection}=[blue,pattern=north east lines,pattern color=blue];
%\tikzstyle{full-self}=[orange,pattern=north west lines,pattern color=orange];
%\tikzstyle{full-market}=[red,pattern=crosshatch dots,pattern color=red];
%
%\tikzstyle{partial-protection}=[blue,pattern=north east lines,pattern color=blue,opacity=.4];
%\tikzstyle{partial-self}=[orange,pattern=north west lines,pattern color=orange,opacity=.4];
%\tikzstyle{partial-market}=[red,pattern=crosshatch dots,pattern color=red,opacity=.4];
%
%\tikzstyle{partial-all}=[gray,pattern=grid,pattern color=gray,opacity=.4];
%
%\tikzstyle{passivity}=[dashed,thick];
%\tikzstyle{special}=[red,thick];
%
%\begin{tikzpicture}
%
%	%	Legend
%	
%%	\draw [special] (0,-.5) rectangle ++(.25,.25) ++(0,-.125) node [right=2mm,black,opacity=1] {partial protection and full market insurance};
%
%	\draw [partial-self,line width=.8mm,dash pattern=on 1mm off 1mm] (-.2,-.875)--++(.6,0) node [right=0mm,black,opacity=1] {area 6};
%	\draw [partial-market,line width=.8mm,dash pattern=on 1mm off 1mm] (.4,-.875)--++(-.6,0);
%
%	\draw [partial-protection,line width=.6mm] (-.2,-1.375)--++(.6,0) node [right=0mm,black,opacity=1] {area 5};
%	
%	\draw [partial-all] (0,-.5) rectangle ++(.25,.25) ++(0,-.125) node [right=2mm,black,opacity=1] {area 4};
%	\draw [full-market] (0,0) rectangle ++(.25,.25) ++(0,-.125) node [right=2mm,black,opacity=1] {area 3};
%	\draw [full-self] (0,.5) rectangle ++(.25,.25) ++(0,-.125) node [right=2mm,black,opacity=1] {area 2};
%	\draw [full-protection] (0,1) rectangle ++(.25,.25) ++(0,-.125) node [right=2mm,black,opacity=1] {area 1};
%\end{tikzpicture}
%
%\begin{tikzpicture}[>=stealth,scale=7]
%
%%	\draw [partial-all,draw=none] (.125,0) rectangle (1,.5);
%%	\draw [full-self,draw=none] (0,0)--(0,.375)--(.625,.375)--(1,0)--(0,0);
%%	\draw [full-protection,draw=none] (0,0) -- (.125,.125) -- (.125,1) -- (0,1) -- (0,0);
%%	% \draw [full-market,draw=none] (.125,.5) rectangle (1,1);
%%	\draw [full-market,draw=none] (0,.5) rectangle (1,1); % combined (1) and (5)
%%	
%%	\draw [partial-market,line width=.8mm,dash pattern=on 1mm off 1mm] (.5,.5) node [below,black] {\small FIX ME}--(1,.5);
%%	\draw [partial-self,line width=.8mm,dash pattern=on 1mm off 1mm] (1,.5)--(.5,.5);
%%	
%%	\draw [partial-protection,line width=.6mm] (.125,.5)--(.125,1);
%	
%%	\draw [special] (0,0) -- (.5,.5) -- (0,.5) -- (0,0);
%	\myaxes{}
%\end{tikzpicture}



Direct benefit

Welfare might mean aggregated direct benefit

Welfare might mean aggregated support


Features we want to capture:

\begin{itemize}
\item Each individual may have an arbitrary initial evaluation of any law based on any reason.
\item Each individual's evaluation may (but need not) be influenced by evaluations of others.
\item Each individual may have a varying level of influence within a given social network.
\item We distinguish an individual's derived benefit of a law from an individual's support of a law. 
\item An individual's support is a function of their derived benefit together with a weighted influence of others. %influence With respect to a fixed law, suppose the mixing process happens and we measure, then we cannot tell the difference between the influence and the case where each individual makes their own evaluation or is influenced by others.  
\item 
\end{itemize}

\subsection{Motivation}
Want to talk about divisiveness.
Divisiveness means strong disagreement which can be measured by standard deviation in preferences.

Want to talk about polarization.
Polarization means grouping of individuals according to extremes.
Wikipedia causes: 
-some argue that comes from elites, or activists,
- some argue it is an illusion caused by a combination of limited choices, a small number of hot button issues, or the exploitation of cultural differences for various ends.

Want to talk about privacy.
Privacy means a right of individuals to have private property that is free from inspection or search. Privacy is a right afforded by msot demcratic societies primarily motivated by preventing abuse.

For each pair of citizens $$i,j\in\{1,\dots N\},$$ define $$a_{ij}\in [0,1]$$ to be the affinity of person $i$ for person $j$, or the extent to which person $i$ takes the well-being of person $j$ into account when making decisions, or the extent to which person $j$ influences the decision of person $i$. For each $i$ we must have $$\sum_ja_{ij}=1.$$%something similarly concrete-ish.

Given a law $L$ and a citizen $j\in\{1,\dots N\}$, %define $$L(j)\in[0,1]$$ to be the probability that the law targets $j$.
%Alternatively, 
define $$V_L(j)\in\mathbb{R}$$ to be the direct impact of the law on person $j$. The direct  impact of $L$ is unbounded because laws may save a life or take away life from individuals impacted by them.

Certainly in many cases, an individual's direct valuation for a law is perfectly in tune with the perceived direct impact on the law.  In such cases, a person's continued support for $L$ might continue to be $V_L(i)$ indefinitely. % weight his or her own opinion so much above others that for all practical purposes we have $a_{ii}\approx 1$, $a_{ij}\approx 0$ for $j\neq i$, and the support of $i$ for $L$ is exactly $$a_{ii}L(i)=L(i).$$

In other cases, for example where a law such as a sodomy ban has little or no direct impact, initial evaluation of the law may be something very close to zero, i.e. neither positive nor negative.  Nevertheless a person's view of the law may evolve based on the views and experiences of their influences.  In such cases, we may suppose that after some amount of time, an individual's support for L evolves to become $$U_L(i)=\sum_ja_{ij}V_L(j).$$

%Every law $L$ induces a function from the set of all citizens to the real line and thus induces a linear ordering of those citizens given by $j_1<j_2$ iff $L(j_1)<L(j_2)$.


%For a fixed citizen $i$ we may consider $$\cdots a_i \cdots$$ as a row vector which expresses the views of $i$ toward every other citizen.  Each row vector is normalized, so that the total influence of all people combined to the views of person $i$ is 100\%. 
%
%We may also consider for a citizen $j$ the column vector $$\begin{tabular}{c}\vdots\\$a_j$\\\vdots\end{tabular}$$ which expresses the views that other citizens have towards them.  Column vectors may have arbitrary norms, reflecting the notion that certain individuals carry more influence than others.
%
%Note that in a fully independent and unbiased and rational society we would have $a_{ii} =1$  and $a_{ij}=0$ for every $i\neq j$. That is, every person would make decisions on the basis of the direct impact of the law on their own wellbeing.
%
%In this simple case, we would necessarily have $a_{ii}=L(i)$.  That is, each citizen would support a law in proportion to the individual effect that the law has on them
%
%In real-world societies for which citizens are not completely independent -- for example, if they are more socialist, or more polarized, the support of a law also depends on the extent to which it impacts those whose views they are inclined to follow. 
%
%% other citizens both positively and negatively.
%
%The extent to which citizen $i$ supports the law is given by $$\sum_{j}a_{ij}L(j).$$
%Extrapolating from the simple case, let us say that the extent to which citizen $i$ supports a law $L$ is proportional to $$\left<a_i,L\right>$$
%
%Another way to consider this is that the matrix multiplication $$[a_{ij}]L$$ gives a column vector whose $i^{th}$ entry describes the extent to which citizen $i$ supports $L$.

If we order the elements of this vector and graph it on a line, we can see directly from the graph a number of things
\begin{enumerate}
\item what percentage supports the law -- aka where the graph crosses the horizontal axis
\item the existence of distinct homogeneous clusters -- regions along the horizontal axis where the function takes on a homogeneous value
\item the extent to which the law is divisive for the society -- aka the magnitude of disutility for non-supporters of the law.
\end{enumerate}

How does privacy enter this picture?
Privacy decreases the law's overall enforceability, and in particular, it makes it harder to enforce the law selectively.

The crux of the analysis will be to show that privacy is especially welfare-enhancing in cases where a society is more polarized.
%!TEX root = paper.tex
%analysis.tex

\section{Analysis}
\label{sec:analysis}

In this section, we specify a functional form for $L_0$ in order to discuss the effects of privacy.  While this reduces the generality of our model, it allows us to distinguish between different notions of privacy protection.  

We imagine that three conditions must apply for citizen $i$ to be punished by a law:

\begin{enumerate}
\item As the text of the law is written, it specifies that citizen $i$ is engaged in unlawful behavior.  We assume this occurs with probability $Cr(i)$.
\item Citizen $i$ is searched by the authorities.  We assume this occurs with probability $S(i)$.
\item A search on citizen $i$ is successful in finding evidence.  Conditional on the previous conditions, we assume this occurs with probability $F(i)$.  For simplicity, we will assume this function is a constant and write it as $F$.
\end{enumerate}

The probability that citizen $i$ is punished is therefore $S(i)Cr(i)F$.  The total number of citizens punished is given by $T = \sum_i S(i)Cr(i)F$.  We let $L_0$ take the following form:

$$L_0(i) = BT - S(i)Cr(i)FP$$

Each citizen earns a benefit from the law, proportional to the number of people that are caught, but earns disutility P if she is punished.

Our model allows us to compare four types of privacy, two of which are technological in nature, and two of which are legal.

\subsection{Attribute Privacy}

Attribute privacy is the notion that a person can conceal personal characteristics, which authorities may use to identify them as someone likely to commit a crime.  In the extreme case, individuals may be fully anonymized, which we express by saying that $S(i)$ is a constant. 

$$S(i)=S$$ In less extreme cases, we may place restrictions on $S(i)$.

For a given text of a law, which specifies $Cr(i)$, restrictions on $S(i)$ may prevent a majority from emerging to support the law.  This is a simple consequence of the fact that restrictions on $S(i)$ translate into restrictions on $L_0(i)$ and $L_1(i)$.  In practice, we may expect authorities to restrict searches to a minority of the population, thereby ensuring that $L_0(i)$ is positive for a majority of individuals.

An example in which individuals are fully or nearly anonymized is the case in which users of TOR hide the online activities.  We may specify this case by saying that $S(i)$ is a constant.  In this case, the shape of $L_0$ is entirely determined by $Cr(i)$, up to a scaling factor.  The number of individuals that support the law depends only on $Cr(i)$ and how polarized society is.

\subsection{Search Privacy}

Search Privacy is the idea that a citizen may use technology to prevent the discovery of evidence in the event that she is searched.  We represent this as a decrease in the parameter, $F$, 
$$F<\lambda$$
representing the chance that evidence is found when a citizen breaking the law is searched.  In our specification, $F$ appears in both components of $L_0$ so this scales $L_0(i)$ equally for all $i$.  This also implies that $L_1(i)$ is scaled equally for all $i$, which means that the number of citizens supporting the law is unaffected.

Although search privacy will not disrupt a majority of individuals that support a law, it will affect welfare, scaling it towards zero.  Search privacy may therefore be welfare benefiting in the case of divisive laws for which total welfare is decreased.

\subsection{Search Quantity Privacy}

Our first legal notion of privacy corresponds to the idea that searches should not be widespread in a society.  This is exemplified by a recent case before the US supreme court, Utah v. Strieff.  This case revolves around the extent to which the Fourth Amendment restricts police from searching citizens.  Detective Douglas Fackrell stopped Edward Strieff outside a house he was surveilling and asked for identification.  When he discovered that Strieff had an outstanding warrant for a minor traffic violation, he proceeded to search him and cound methamphetamine in Strieff's pockets.  According to the so-called exclusionary rule, evidence gained in violation of the Fourth Amendment is generally not permitted in court, but it is unclear whether the rule applies when a warrant is found after an improper detainment.  During oral argument, justice Sonya Sotomayor asked ``what stops us from becoming a police state and just having the police stand on the corner down here and stop every person, ask them for identification, put it through? and if a warrant comes up, searching them?"

At the core of these arguments lies a notion that searches should be rare in society rather than widespread.  We encode this by requiring that the total number of searches is less than some bound, $$\sum_i S(i) < M$$.  Laws that don't meet this requirement may be declared unenforceable.

Search Quantity Privacy prevents the enforcement of laws against a large fraction of a population.


\subsection{Search Specificity Privacy}

Another notion of privacy supposes that authorities must have individualized reasonable suspicion to conduct a search.  This is similar to the notion of Search Quantity Privacy, defined above, but the focus is not on the total number of searches, but rather on how well-targeted the searches are.  We may encode this by requiring that a certain fraction of searches result in the finding of evidence,

$$\frac{\sum_i S(i)Cr(i)F}{\sum_i S(i)} > \rho$$



(note: should consider divisive laws in particular...)


%!TEX root = paper.tex
%results.tex

%\section{Numerical Illustrations}
%\label{sec:results}
%



%
%The shaded region in Figure \ref{fig:surplus2} shows the total consumer surplus in equilibrium when $k=1$.
%
%\begin{figure}
%\begin{center}
%\includegraphics[width=0.75\textwidth]{plots/figure3}
%\caption{figure 3}
%\label{fig:surplus2}
%\end{center}
%\end{figure}
%

%%
% $Id: numerical.tex 686 2010-02-22 19:46:41Z nicolasc $
%
\section{Numerical evaluation}
\label{sec:numerical}

\begin{figure}
\begin{center}
\includegraphics[width=0.75\textwidth]{plots/gamma}
\caption{\label{fig:gamma} {\bf Gamma -- clever caption here.} Explanation of the clever caption.}
\end{center}
\end{figure}
\begin{figure}
\begin{center}
\includegraphics[width=0.75\textwidth]{plots/bs}
\caption{\label{fig:bs} {\bf Best shot.} Explanation of the clever caption.}
\end{center}
\end{figure}
\begin{figure}
\begin{center}
\shortstack{
        \includegraphics[width=0.75\textwidth]{plots/wl}\\
        {\footnotesize (a) Overall trends.}
}
\quad
\shortstack{
        \includegraphics[width=0.45\textwidth]{plots/wl2}\\
        {\footnotesize (b) Zoom on small values of the protection cost $b$.}
}
\caption{\label{fig:wl} {\bf Weakest link.} Explanation of the clever caption.}
\end{center}
\end{figure}
\begin{figure}
\begin{center}
\includegraphics[width=0.75\textwidth]{plots/te}
\caption{\label{fig:te} {\bf Total effort.} Explanation of the clever caption.}
\end{center}
\end{figure}

%!TEX root = paper.tex
%concl.tex

\section{Discussion and Conclusion}
\label{sec:concl}

Our study is an attempt to understand privacy, not at the level of individual incentives, but at the level of communities and the relationships they have to each other and to the state.  Though our modeling framework is exploratory, it reveals some of the complexity inherent in these relationships.  We were able to relate privacy to polarization and the divisiveness of laws, but found that outcomes depend critically on what notion of privacy is in play.  Privacy enforced through technology can have dramatic effects on what laws are enforced, but all laws may be rendered ineffective whether divisive or not.  A privacy technology may work, after all, whether it is concealing sexual behavior between consenting adults, or a plot to assassinate a leader in government.  Legal notions of privacy hold the promise of more precise judgements.  Courts can identify disadvantaged groups and extend protections to them, without extending those same protections to say, serial killers.  Yet our model predicts that here too, privacy is not perfectly tailored to prevent the enforcement of divisive laws.  Further research is needed to assess how privacy laws may work in conjunction with anti-discrimination laws and policies, to better protect marginalized groups. 

In the future, we plan to extend our model to capture more features of the legal system and of privacy protection.  One important addition will be a description of how authorities gather information on potential law breakers.  A detailed model might describe citizens with a collection of attributes, each of which may carry information about a citizen's propensity to violate a specific law.  Some attributes may be hidden with technological privacy measures, some may be the topic of legal protection, and others may remain public and available to police as they direct their investigations.  

We would further like to model the issues that arise when police have the right to search a large number of people, but only enough resources to choose a small number.  The ability to arbitrarily select which citizens to search carries a significant amount of power, even when the number of searches remains small.  A more complete model would separate legality from the actual performance of a search in order to highlight these issues.

As technology advances, many established notions of privacy face considerable pressure to evolve.  Location monitoring, deep packet inspection, linking of consumer databases, and face recognition are just a few of the threats to our ability to control our personal information.  We hope that studies like ours will help spur discussion about the role privacy plays in maintaining balances of power, and how future definitions of privacy may best be structured to protect vulnerable groups and individuals.



\bibliographystyle{plain}
\bibliography{local,references}

\end{document}
% ___END___
