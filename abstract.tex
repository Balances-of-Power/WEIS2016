%!TEX root = paper.tex
%abstract.tex
%
Notions of privacy are particularly salient to marginalized groups of people, especially when they find themselves disproportionately affected by the enforcement of laws.  We use game theoretic modeling to explore the connections between privacy, polarization, and the divisiveness of laws.  Our framework is based on a population of citizens that may be more or less polarized.  A law is defined in terms of its effect on each citizen and must gain support from a majority in order to pass.  We define a notion of divisiveness which allows us to measure the extent to which a law disproportionately affects different groups of citizens.  Our framework allows us to explore four distinct notions of privacy, two that result from technological measures and two that emerge from legal theory.  We find that privacy can prevent the passage of certain divisive laws, but the effects depend strongly on which type of privacy is in use.

