%!TEX root = paper.tex
%related.tex

\section{Related work}
\label{sec:related}

Related work

Privacy on the ground. Hard to find citation in bibtex. Can cite a related journal article instead?

Caviar and Yachts \cite{johnsoncaviar}

Government Surveillance \cite{laskowskigovernment}

Journal article on polariztion \cite{poole1984polarization}

Book on Authoritarianism \cite{hetherington2009authoritarianism}

 most voters have been
and remain overwhelmingly moderate in their policy positions -- cites half a dozen primary sources -- Article on causes of polarization \cite{layman2006party}

Causes and consequences of polarization \cite{barber2015causes}

noisy signals example is informational cascades, people receive a noisy signal and rely on friends and colleagues to distill the essential parts
noisy information signal combined with bayesian updating. 



The notion that people exhibit herding behaviour in predictable circumstances has been around for decades. \cite{shiller1995conversation}

\benjamin{this is plagiarized, must rewrite substantially}
The adoption rate of drought-resistant hybrid seed corn during the Great Depression and Dust Bowl was slow despite its significant improvement over the previously available seed corn. Researchers at Iowa State University were interested in understanding the public's hesitation to the adoption of this significantly improved technology. After conducting 259 interviews with farmers \cite{carboneau2005using} it was observed that the slow rate of adoption was due to how the farmers valued the opinion of their friends and neighbors instead of the word of a salesman. See \cite{beal1957diffusion} for the original report.


\cite{bikhchandani1992theory}
``An information cascade occurs when it is optimal for an individual, having observed the actions of those ahead of him, to follow the begvior of the preceeding individual without regard to his own information."



McCarty, Poole, and Rosenthal (2006) demonstrated a close correlation between economic
inequality and polarization in the United States


polarized electorate, southern realignment, Gerrymandering, primary elections, economic inequality, money in politics, media environment, congress-based factors such as congressional rule changes, majority party agenda control, party pressures, teamsmanship, breakdown of bipartisan norms.  All of these are discussed in  \cite{poole1984polarization}