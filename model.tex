%!TEX root = paper.tex
%model.tex

\section{Model}
\label{sec:model} 

Features we want to capture:

\begin{itemize}
\item Each individual may have an arbitrary initial evaluation of any law based on any reason.
\item Each individual's evaluation may (but need not) be influenced by evaluations of others.
\item Each individual may have a varying level of influence within a given social network.
\item We distinguish an individual's derived benefit of a law from an individual's support of a law. 
\item An individual's support is supplemented by a weighted influence of others. %influence With respect to a fixed law, suppose the mixing process happens and we measure, then we cannot tell the difference between the influence and the case where each individual makes their own evaluation or is influenced by others.  
\item 
\end{itemize}

For each pair of citizens $$i,j\in\{1,\dots N\},$$ define $$a_{ij}\in [0,1]$$ to be the affinity of person $i$ for person $j$, or the extent to which person $i$ takes the well-being of person $j$ into account when making decisions, or the extent to which person $j$ influences the decision of person $i$. For each $i$ we must have $$\sum_ja_{ij}=1.$$%something similarly concrete-ish.

Given a law $L$ and a citizen $j\in\{1,\dots N\}$, %define $$L(j)\in[0,1]$$ to be the probability that the law targets $j$.
%Alternatively, 
define $$L_0(j)\in\mathbb{R}$$ to be the direct impact of the law on person $j$. The impact of $L$ is unbounded because laws may save a life or take away life from individuals impacted by them.

Certainly in many cases, an individual's support for a law is perfectly in tune with the perceived direct impact on the law.  In such cases, a person's continued support for $L$ might continue to be $L_0(i)$ indefinitely. % weight his or her own opinion so much above others that for all practical purposes we have $a_{ii}\approx 1$, $a_{ij}\approx 0$ for $j\neq i$, and the support of $i$ for $L$ is exactly $$a_{ii}L(i)=L(i).$$

In other cases, for example where a law such as a sodomy ban has little or no direct impact, initial evaluation of the law may be something very close to zero, i.e. neither positive nor negative, but a person's view of the law may evolve based on the views and experiences of their influences.  In such cases, we may suppose that after some amount of time, an individual's support for L evolves to become $$L_1(i)=\sum_ja_{ij}L_0(j).$$

%Every law $L$ induces a function from the set of all citizens to the real line and thus induces a linear ordering of those citizens given by $j_1<j_2$ iff $L(j_1)<L(j_2)$.


%For a fixed citizen $i$ we may consider $$\cdots a_i \cdots$$ as a row vector which expresses the views of $i$ toward every other citizen.  Each row vector is normalized, so that the total influence of all people combined to the views of person $i$ is 100\%. 
%
%We may also consider for a citizen $j$ the column vector $$\begin{tabular}{c}\vdots\\$a_j$\\\vdots\end{tabular}$$ which expresses the views that other citizens have towards them.  Column vectors may have arbitrary norms, reflecting the notion that certain individuals carry more influence than others.
%
%Note that in a fully independent and unbiased and rational society we would have $a_{ii} =1$  and $a_{ij}=0$ for every $i\neq j$. That is, every person would make decisions on the basis of the direct impact of the law on their own wellbeing.
%
%In this simple case, we would necessarily have $a_{ii}=L(i)$.  That is, each citizen would support a law in proportion to the individual effect that the law has on them
%
%In real-world societies for which citizens are not completely independent -- for example, if they are more socialist, or more polarized, the support of a law also depends on the extent to which it impacts those whose views they are inclined to follow. 
%
%% other citizens both positively and negatively.
%
%The extent to which citizen $i$ supports the law is given by $$\sum_{j}a_{ij}L(j).$$
%Extrapolating from the simple case, let us say that the extent to which citizen $i$ supports a law $L$ is proportional to $$\left<a_i,L\right>$$
%
%Another way to consider this is that the matrix multiplication $$[a_{ij}]L$$ gives a column vector whose $i^{th}$ entry describes the extent to which citizen $i$ supports $L$.

If we order the elements of this vector and graph it on a line, we can see directly from the graph a number of things
\begin{enumerate}
\item what percentage supports the law
\item the extent to which the law is divisive for the society
\end{enumerate}

How does privacy enter this picture?
Privacy decreases the law's overall enforceability, and in particular, it makes it harder to enforce the law selectively.

The crux of the analysis will be to show that privacy is especially welfare-enhancing in cases where a society is more polarized.