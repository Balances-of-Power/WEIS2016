%!TEX root = paper.tex
%model.tex

\section{Model}
\label{sec:model} 

For each pair of citizens $$i,j\in\{1,\dots N\},$$ define $$a_{ij}\in [-1,1]$$ to be the affinity of person $i$ for person $j$, or the extent to which person $i$ takes the well-being of person $j$ into account when making decisions, or something similarly concrete-ish.

Given a law $L$ and a citizen $j\in\{1,\dots N\}$, define $$L(j)\in[0,1]$$ to be the probability that the law targets $j$.

Every law $L$ induces a function from the set of all citizens to the interval $[0,1]$ and thus induces a linear ordering of those citizens given by $j_1<j_2$ iff $L(j_1)<L(j_2)$.


For a fixed citizen $i$ we may consider $$\cdots a_i \cdots$$ as a row vector which expresses the views of $i$ toward every other citizen.
We may also consider for a citizen $j$ the column vector $$\begin{tabular}{c}\vdots\\$a_j$\\\vdots\end{tabular}$$ which expresses the views that other citizens have towards them.
Note that in a fully independent and unbiased society we would have $a_{ii} =1$  for every $i$ and $a_{ij}=0$ for every $i\neq j$.  In this simple case, we would necessarily have $a_{ii}=L(i)$.  That is, each citizen would support a law in proportion to the individual effect that the law has on them

In real-world societies for which citizens are not completely independent -- for example, if they are more socialist, or more polarized, the support of a law also depends on the extent to which it targets other citizens.

Extrapolating from the simple case, let us say that the extent to which citizen $i$ supports a law $L$ is proportional to $$\left<a_i,L\right>$$

Another way to consider this is that the matrix multiplication $$[a_{ij}]L$$ gives a column vector whose $i^{th}$ entry describes the extent to which citizen $i$ supports $L$.

If we order the elements of this vector and graph it on a line, we can see directly from the graph a number of things
\begin{enumerate}
\item what percentage supports the law
\item the extent to which the law is divisive for the society
\end{enumerate}

How does privacy enter this picture?
Privacy decreases the law's overall enforceability, and in particular, it makes it harder to enforce the law selectively.

The crux of the analysis will be to show that privacy is especially welfare-enhancing in cases where a society is more polarized.