%!TEX root = paper.tex
%intro.tex
%

\section{Introduction}
\label{sec:intro}




In 2006, Utah police received an anonymous tip about drugs being sold out of a house in South Salt Lake. Detective Douglas Fackrell spent several hours surveilling the house in an unmarked car, but saw only modestly suspicious activity.  After a week, he stopped an individual, Edward Strieff, as he was exiting the house and asked for identification.  When he discovered that Strieff had an outstanding warrant for a minor traffic violation, Detective Fackrell proceeded to search him and found methamphetamine in Strieff's pockets.  

This event marked the start of legal battle that reached the US Supreme Court this year.  Although a warrant generally provides adequate reason for a search, the police conceded that Fackrell did not have reasonable suspicion to detain Strieff in the first place.  The supreme court has long held that evidence gained in violation of the Fourth Amendment is inadmissible as evidence - the so-called exclusionary rule.  The case revolves around the extent to which the exclusionary rule protects the privacy of citizens when new information, such as an arrest warrant, arrives after an improper detainment.

Utah v. Strieff reveals the extent to which our value of privacy is bound up in notions of power and polarization.  During oral argument in February, Justice Sonya Sotomayor posed the following question. 

\begin{quote}``What stops us from becoming a police state and just having the police stand on the corner down here and stop every person, ask them for identification, put it through, and if a warrant comes up, searching them?"
\end{quote}

At the heart of this argument is a recognition that privacy protects entire groups of people.  One doesn't have to have drugs in their pocket to object to arbitrary searches by police.  Privacy places limitations on police power, affecting the playing field faced by all citizens.  Even when the vast majority of police officers abide by strict ethical standards, the prospect of running into a corrupt one remains threatening.  Furthermore, the ability to invade the privacy of citizens has been argued to increase incentives for governments to abuse their power. 

Calls for privacy are particularly acute when a particular group is disadvantaged or marginalized.  In this case, the laws that police enforce may disproportionally affect the marginalized group.  A classic example is the disparity between sentences for powder cocaine, typically associated with rich white communities, and crack cocaine, which is associated with disadvantaged black communities.  Before the fair sentencing act of 2010, the weight of powder cocaine needed to trigger certain federal criminal penalties was 100 times greater than the weight of crack cocaine that would trigger the same penalties.  The disparity is said to be a significant factor behind the large number of African Americans that are sentenced for drug offenses.

To understand how laws and police enforcement affect disadvantaged groups, we must also understand how society is polarized between different groups of people to begin with.  How are laws passed that benefit one group at the expense of another group.  Moreover, can privacy protection help marginalized groups overcome their disadvantageous position?

In this paper, we use game theoretic modeling to explore the connections between privacy, polarization, and the passage of divisive laws.  Our framework is based on a population of citizens that influence what laws are passed, or what laws are maintained.  A law may affect each individual in a different way.  Our framework allows citizens to only worry about the

describe the way in which people decide whether or not to support a particular law.  While people may just consider their 

Laws that are supported by a majority of citizens pass or are maintained.  Although the democratic process involves many factors before gathering a simple majority, we believe this is a useful and tractable way to explain the types of laws that exist in society.

We define a notion of polarization, and consider this as an independent variable in our model.  Individual laws are described according to their effect on each citizen.  We further define a notion of divisiveness, which allows us to measure the extent to which a law disproportionately affects different groups of citizens.



Until we resolve our society's problems with irrational divisiveness, we will work on privacy, which mitigates the problems of laws in the context of actual divisiveness.

Discussion examples
\begin{itemize}
\item Marijuana and drug laws
\item Income redistribution
\item Fergeuson style searches
\item stop and frisk
\item prohibition
\item iphone data
\item Muslim databases
\end{itemize}