%!TEX root = paper.tex
%analysis.tex

\section{Analysis}
\label{sec:analysis}




In this section, we specify a functional form for $L_0$ in order to discuss the effects of privacy.  While this reduces the generality of our model, it allows us to distinguish between different notions of privacy protection.  

We imagine that three conditions must apply for citizen $i$ to be punished by a law:

\begin{enumerate}
\item As the text of the law is written, it specifies that citizen $i$ is engaged in unlawful behavior.  We assume this occurs with probability $Cr(i)$.
\item Citizen $i$ is searched by the authorities.  We assume this occurs with probability $S(i)$.
\item A search on citizen $i$ is successful in finding evidence.  Conditional on the previous conditions, we assume this occurs with probability $F(i)$.  For simplicity, we will assume this function is a constant and write it as $F$.
\end{enumerate}

The probability that citizen $i$ is punished is therefore $S(i)Cr(i)F$.  The total number of citizens punished is given by $T = \sum_i S(i)Cr(i)F$.  We let $L_0$ take the following form:

$$L_0(i) = BT - S(i)Cr(i)FP$$

Each citizen earns a benefit from the law, proportional to the number of people that are caught, but earns disutility P if she is punished.

Our model allows us to compare four types of privacy, two of which are technological in nature, and two of which are legal.

\subsection{Attribute Privacy}

Attribute privacy is the notion that a person can conceal personal characteristics, which authorities may use to identify them as someone likely to commit a crime.  In the extreme case, individuals may be fully anonymized, which we express by saying that $S(i)$ is a constant. 

$$S(i)=S$$ In less extreme cases, we may place restrictions on $S(i)$.

For a given text of a law, which specifies $Cr(i)$, restrictions on $S(i)$ may prevent a majority from emerging to support the law.  This is a simple consequence of the fact that restrictions on $S(i)$ translate into restrictions on $L_0(i)$ and $L_1(i)$.  In practice, we may expect authorities to restrict searches to a minority of the population, thereby ensuring that $L_0(i)$ is positive for a majority of individuals.

An example in which individuals are fully or nearly anonymized is the case in which users of TOR hide the online activities.  We may specify this case by saying that $S(i)$ is a constant.  In this case, the shape of $L_0$ is entirely determined by $Cr(i)$, up to a scaling factor.  The number of individuals that support the law depends only on $Cr(i)$ and how polarized society is.

\subsection{Search Privacy}

Search Privacy is the idea that a citizen may use technology to prevent the discovery of evidence in the event that she is searched.  We represent this as a decrease in the parameter, $F$, 
$$F<\lambda$$
representing the chance that evidence is found when a citizen breaking the law is searched.  In our specification, $F$ appears in both components of $L_0$ so this scales $L_0(i)$ equally for all $i$.  This also implies that $L_1(i)$ is scaled equally for all $i$, which means that the number of citizens supporting the law is unaffected.

Although search privacy will not disrupt a majority of individuals that support a law, it will affect welfare, scaling it towards zero.  Search privacy may therefore be welfare benefiting in the case of divisive laws for which total welfare is decreased.

\subsection{Search Quantity Privacy}

Our first legal notion of privacy corresponds to the idea that searches should not be widespread in a society.  This is exemplified by a recent case before the US supreme court, Utah v. Strieff.  This case revolves around the extent to which the Fourth Amendment restricts police from searching citizens.  Detective Douglas Fackrell stopped Edward Strieff outside a house he was surveilling and asked for identification.  When he discovered that Strieff had an outstanding warrant for a minor traffic violation, he proceeded to search him and cound methamphetamine in Strieff's pockets.  According to the so-called exclusionary rule, evidence gained in violation of the Fourth Amendment is generally not permitted in court, but it is unclear whether the rule applies when a warrant is found after an improper detainment.  During oral argument, justice Sonya Sotomayor asked ``what stops us from becoming a police state and just having the police stand on the corner down here and stop every person, ask them for identification, put it through? and if a warrant comes up, searching them?"

At the core of these arguments lies a notion that searches should be rare in society rather than widespread.  We encode this by requiring that the total number of searches is less than some bound, $$\sum_i S(i) < M$$.  Laws that don't meet this requirement may be declared unenforceable.

Search Quantity Privacy prevents the enforcement of laws against a large fraction of a population.


\subsection{Search Specificity Privacy}

Another notion of privacy supposes that authorities must have individualized reasonable suspicion to conduct a search.  This is similar to the notion of Search Quantity Privacy, defined above, but the focus is not on the total number of searches, but rather on how well-targeted the searches are.  We may encode this by requiring that a certain fraction of searches result in the finding of evidence,

$$\frac{\sum_i S(i)Cr(i)F}{\sum_i S(i)} > \rho$$




\section{A Two-Population Model}

To further explore issues of privacy, we must assume a particular structure for consumer affinity.  In this section, we analyze the special case that there are two types of citizens, labeled 0 and 1.  Type 0 citizens are assumed to form the majority with population $n_0$.  Type 1 citizens form the minority with population $n_1 < n_0$.  Within each group, we assume that all citizens have identical preferences.  The affinity of a citizen of type $i$ for a citizen of type $j$ is labeled $a_ij$.  For exposition, we make the additional assumption that $a_{00} > a_{01}$ and $a_{11} > a_{10}$, meaning that citizens care more about what happens to an individual of their own type than one of the other type.  This also implies that 
\begin{align}
a_{00} > 1/(n_0 + n_1) > a_{01} \\
a_{11} > 1/(n_0 + n_1) >  a_{10}
\end{align}

Given a law $(Cr, S, F)$, let $p_0 = Cr(0)S(0)F$ be the fraction of the majority that are punished, and let $p_1 = Cr(1)S(1)F$ be the fraction of the minority that are punished.  We use Figure \paul{blah} to plot these quantities for different possible laws.  On this graph, the x-axis represents $p_0$ and the y-axis represents $p_1$.  In our model, the text of the law specifies the maximum fraction of each type of citizen that can be punished, $Cr(0)$ and $Cr(0)$, which occurs when $F=1$ and everyone is searched.  This region is represented by the outermost rectangle in the Figure, labeled A.

\begin{tikzpicture}[scale = 6]

       % x axis
\draw [->] (0, 0) -- (1.1, 0) node [below right] {$p_0$};
\draw [shift={(1.0, 0)}] (0,0.02) -- (0, -0.02) node[below] {$Cr(0)$};

       % y axis
\draw [->] (0, 0) -- (0, 1.3) node [above left] {$p_1$};
\draw [shift={(0, 1.2)}] (-0.02,0) -- (0.02,0) node[shift={(-0.8,0)}] {$Cr(1)$};

	%diagonal
\draw [->,dashed] (-0.1, -0.1) -- (1.2, 1.1) node [above] {$B$};

	%attribute privacy line 
\draw [->,dashed] (0, 0) -- (1.1, 1.32) node [above] {$C$};

	%right
\draw [-,dashed] (1, 0) -- (1, 1.2);

	%top
\draw [-,dashed] (0, 1.2) -- (1, 1.2); 

	%point A
\draw[shift={(0.2, 1.2)}] node [above] {$A$};

	%right of search privacy
\draw [-,dashed] (.5, 0) -- (.5, .6);

	%top of search privacy
\draw [-,dashed] (0, .6) -- (.5, .6); 

	%point D
\draw[shift={(0.2, .6)}] node [above] {$D$};


\end{tikzpicture}


Since more than half of the population is of type 0, a law may pass whenever the support of each type 0 individual is nonnegative.  We may compute support from individuals of type 0 as follows,

\begin{align}
L_1&(0) = n_0 a_{00}  L_0(0) + n_1 a_{01}  L_0(1) \\
&= \beta t - n_0 a_{00}p_0 P - n_1 a_{01}  p_1 P \\
&= \beta \left[ \frac{n_0 p_0}{n_0 + n_1}   + \frac{n_1 p_1}{n_0 + n_1} \right]  - n_0 a_{00}p_0 P - n_1 a_{01}  p_1 P \\
&= \left[  \frac{\beta n_0}{n_0 + n_1} - n_0 a_{00}P  \right]p_0 + \left[ \frac{\beta n_1}{n_0 + n_1} -  n_1 a_{01}P \right] p_1
\end{align}

Letting $b_{00} = \frac{\beta }{n_0 + n_1} -  a_{00}P $ and $b_{01} = \frac{\beta}{n_0 + n_1} -   a_{01}P $, we can write,

\begin{align}
L_1(0) = b_{00}n_0p_0 + b_{01} n_1p_1 
\end{align}

Our assumption that $a_{00} > a_{01}$ guarantees that $b_{00} < b_{01}$.  This means that there are 3 possibilities for the signs of the coefficients on $p_0$ and $p_1$.

\begin{enumerate}
\item $\partial L_1(0) / \partial p_0 \leq 0$ and $\partial L_1(0) / \partial p_1 \leq 0$.  This will happen whenever $\beta \leq   a_{01}P(n_0 + n_1)  $, meaning that the benefit from the law is small relative to the size of the punishment multiplied by the affinity of the majority for the minority.
\item $\partial L_1(0) / \partial p_0 \leq 0$ and $\partial L_1(0) / \partial p_1 > 0$.  This will happen whenever $  a_{01}P(n_0 + n_1) < \beta  \leq a_{00}P(n_0 + n_1) $, or for intermediary amounts of benefit.  In this case, the majority will tend to support laws that punish the minority more and the majority less.
\item $\partial L_1(0) / \partial p_0 > 0$ and $\partial L_1(0) / \partial p_1 > 0$.  This will happen whenever $   a_{00}P(n_0 + n_1) < \beta$, or for high amounts of benefit.  In this case, the majority will tend to support laws that punish both groups as much as possible.
\end{enumerate}

The middle possibility, in which majority support increases with $p_1$ but decreases with $p_0$ is of particular interest, since this is the case in which the majority would favor the most divisive treatment of the two groups.

A law will have majority support if and only if $L_1(0) \geq 0$.  This can be written as,

\begin{align}
L_1(0) = b_{00}n_0p_0 + b_{01} n_1p_1 \geq 0
\end{align}

If condition 1 holds above, no law within rectangle A can have majority support, except for the origin.  This means that nobody will be punished and no benefit results.  If condition 3 holds above, every law within rectangle A has majority support, so a law may pass for any values of $S(0)$, $S(1)$, and $F$.  For the more interesting condition 2, the previous equation divides rectangle A into two regions.  This is depicted by line B in Figure \paul{blah}.  Every law that falls on this line or above can have majority support.

Using a similar argument to that above, we can derive the support of a type 1 citizen for the law to be

\begin{align}
L_1(1) =  b_{10} n_0p_0 + b_{11}n_1p_1
\end{align}

where $b_{10} = \frac{\beta}{n_0 + n_1} -   a_{10}P$ and $b_{11} = \frac{\beta }{n_0 + n_1} -  a_{11}P $ .  Since we assume $a_{10} < a_{11}$, we have $b_{10} > b_{11}$.

We compute divisiveness as the standard deviation,

\begin{align}
s_L &= \sqrt{var(L_1)}  =  \sqrt{E\left[ \left(L_1 - \bar{L} \right)^2 \right] }  \\
&= \sqrt{ \frac{ n_0\left(L_1(0) - \bar{L}\right)^2 + n_1\left(L_1(1) - \bar{L} \right)^2 }{n_0+n_1}  }
\end{align}

where $\bar{L} = \frac{n_0 L_1(0) + n_1L_1(1) }{n_0 + n_1}$ is the average support for the law.  After some algebra, this can be simplified to the following,

\begin{align}
s_L &= \frac{\sqrt{n_0n_1}}{n_0 + n_1} \left| L_1(0) - L_1(1) \right|
\end{align}

In the region of laws that have majority support, it can be shows that $L_1(0) > L_1(1)$. Substituting for $L_1(0)$ and $L_1(1)$, we have

\begin{align}
s_L &= \frac{\sqrt{n_0n_1}}{n_0 + n_1} \left(  b_{00}n_0p_0 + b_{01} n_1p_1 - b_{10} n_0p_0 - b_{11}n_1p_1 \right) \\
&= \frac{\sqrt{n_0n_1}}{n_0 + n_1} \left(  ( b_{00} - b_{10} )n_0p_0 + ( b_{01} - b_{11})n_1p_1 \right) \\
&= \frac{\sqrt{n_0n_1}}{n_0 + n_1} \left(  ( a_{10} - a_{00} )P n_0p_0 + ( a_{11} - a_{01})P n_1p_1 \right) \\
\end{align}

by the ordering on our affinities, we know $a_{10} - a_{00}  < 0$ and $a_{11} - a_{01} > 0$ so

\begin{align}
\partial s_L / \partial p_0 <0 \\
\partial s_L / \partial p_1 >0
\end{align}

In other words, in the region of laws with majority support, divisiveness goes up with the number of minority members punished and down with the number of minority members punished.  This means that the law most favored by the majority, which is in the upper left corner of rectangle A, is also the most divisive.

\subsection{Effects of Attribute Privacy}

Having outlined some basic properties of our model, we turn our attention to the effects of different types of privacy.  We begin with attribute privacy, which we understand to mean restrictions on S(0).  We model the extreme case of anonymity, represented as $S(0) = S(1)$.  The set of possible laws is represented by line C in Figure \paul{blah}.  Note that line C passes through the point $ \left(Cr(0), Cr(1) \right)$ at the upper right of rectangle A.  A more moderate version of attribute privacy may place bounds on the ratio of $S(1)$ to $S(0)$.  For example, we may specify,

\begin{align}
0 < \underbar r < S(1) /S(0) < \bar r < \infty
\end{align}

Since divisiveness is maximized at the point $ \left(0, Cr(1) \right)$, attribute privacy necessarily reduces divisiveness since it disallows laws at this point.  As the figure is depicted, the slope of line C is greater than the slope of line B, 

\begin{align}
Cr(1)/Cr(0) > - b_{00}n_0 / b_{01}n_1 
\end{align}

This means that any law that falls inside line C and rectangle A will also have majority support.  The majority would offer the most support to laws that fall further to the upper right of the line.  It is also possible, for the slope of line C to be less than the slope of line B,

\begin{align}
Cr(1)/Cr(0) > - b_{00}n_0 / b_{01}n_1 
\end{align}

In this case, no law on line C will have majority support except for a law at the origin.  Nevertheless, divisiveness decreases since having no law or a law at the origin brings divisiveness to zero.


